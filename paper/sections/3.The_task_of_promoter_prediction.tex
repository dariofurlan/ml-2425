\section{The Task of Promoter Prediction}

One of the key tasks explored in the DNABERT project is \textbf{promoter prediction}: that is, identifying whether a given DNA sequence contains a promoter region. This is a fundamental step in understanding how genes are regulated.

\subsection{Why promoter prediction matters}

As explained earlier, promoters are short regions of DNA that signal the start of a gene. They are not genes themselves, but they play a crucial role in deciding \textbf{when}, \textbf{where}, and \textbf{how much} a gene is expressed. Misregulation of promoters is often linked to diseases, including cancer, because incorrect gene activation can lead to malfunctioning cells.

Automatically detecting promoters in DNA sequences is important for:
\begin{itemize}
\item Mapping the regulatory landscape of the genome,
\item Understanding gene expression in different cell types,
\item Supporting medical research and diagnostics.
\end{itemize}

However, recognizing promoters is not easy. Promoters do not follow a simple fixed pattern. While some contain recognizable elements like the TATA box, others rely on more complex or subtle sequence signals.

\subsection{How DNABERT tackles the problem}

DNABERT is trained to recognize promoters by learning from examples. During fine-tuning, it is given sequences labeled as either \textit{containing a promoter} (positive) or \textit{not containing a promoter} (negative). The model then learns to classify new sequences based on the patterns it has seen.

Two main evaluation settings are used:

\subsubsection{DNABERT-Prom-300}
This setting focuses on \textbf{proximal promoter regions}: short segments of DNA around the \textit{transcription start site} (TSS), typically from -249 to +50 base pairs relative to the TSS. These regions are critical for transcription to begin.  
In this setup:
\begin{itemize}
\item Positive examples are real promoter sequences.
\item Negative examples are sequences from elsewhere in the genome with similar composition but no promoter activity.
\item The model learns to classify 300 base pair sequences.
\end{itemize}

This setup is well-suited for measuring the model's ability to distinguish true promoter regions from random DNA fragments.

\subsubsection{DNABERT-Prom-scan}
This is a more realistic and challenging scenario. Here, the model receives \textbf{longer DNA sequences} (up to 10,000 base pairs) and must scan them to detect the location of any promoters.  
This task is more difficult because:
\begin{itemize}
\item Promoters are rare in long sequences.
\item The model must avoid false positives while still detecting real promoters.
\end{itemize}

To support this, DNABERT-XL is used: a variant of the model that can process longer input sequences, thanks to architectural adjustments.

\subsection{Types of promoters}

In the original experiments, the promoters were grouped into two categories:
\begin{itemize}
\item \textbf{TATA promoters}: contain a TATA box, a common and easily recognizable pattern.
\item \textbf{No-TATA promoters}: lack the TATA box and are more difficult to identify.
\end{itemize}

This separation helps evaluate how well the model generalizes across different promoter types.

\subsection{Results summary (from DNABERT paper)}

DNABERT significantly outperformed previous models (like CNNs and hybrid CNN-RNNs) in both promoter prediction tasks. It showed:
\begin{itemize}
\item Higher accuracy,
\item Better precision and recall,
\end{itemize}
