\section{The GUE Benchmark}

% \subsection{The GUE Benchmark}

As part of the DNABERT-2 project, the authors introduced a large benchmark called \textit{Genome Understanding Evaluation (GUE)}. This benchmark was designed to better test how well language models can understand and work with real genomic data.

GUE includes \textbf{28 classification tasks} based on DNA sequences from different species and experimental sources. These tasks cover a variety of biological signals, such as:
\begin{itemize}
\item Promoter regions,
\item Transcription factor binding sites (TFBS),
\item Enhancers,
\item Splicing sites,
\item And other regulatory elements.
\end{itemize}

Each task in GUE is treated as a binary classification problem: the model must decide whether a given DNA sequence contains the signal or not.

\subsection{Why GUE matters}

Before GUE, most models\textemdash including DNABERT\textemdash were tested on small, hand-picked datasets, often from a single species (usually human). This made it hard to compare models fairly or evaluate how well they generalize across tasks and species.

GUE solves this by:
\begin{itemize}
\item Providing a \textbf{standardized evaluation} across many biological functions,
\item Including \textbf{multi-species data} to test generalization,
\item Covering both simple and challenging classification tasks.
\end{itemize}

The GUE benchmark was used in the DNABERT-2 paper to show that the new model performs better than previous models (like DNABERT, Enformer, and Nucleotide Transformer) in \textbf{23 out of 28 tasks}, while using fewer parameters and training resources.

\subsection{Scope of our project}

In this project, we focus \textbf{only on the promoter prediction task} one of the classification tasks included in the GUE benchmark. This choice aligns with the original DNABERT paper and allows us to study a well-defined problem with strong biological relevance. Other tasks from GUE are outside the scope of our work.

The specific datasets used in this project are:
\begin{itemize}
    \item \texttt{prom\_300\_all}: Contains human promoter sequences, with and without TATA boxes.
    \item \texttt{prom\_300\_notata}: Contains human promoter sequences without TATA boxes.
    \item \texttt{prom\_300\_tata}: Contains human promoter sequences with TATA boxes.
\end{itemize}
