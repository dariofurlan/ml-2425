
\usepackage[utf8]{inputenc}
\usepackage[a4paper,
            bindingoffset=0.2cm,
            left=1cm,
            right=1cm,
            top=1cm,
            bottom=2cm,
            footskip=1cm]{geometry}

\usepackage{lipsum}
\usepackage[italian,english]{babel}

\usepackage{graphicx} % Required to include images
\usepackage[labelfont=sc]{caption} %DT$18/02/2021 caption style
\usepackage{float} %DT$03/01/2020 better image positioning
\usepackage{cuted} %DT$29/07/2022 allows writing over the whole page

\usepackage{braket} % DT$13/11/2021 qm symbols
\usepackage{enumerate} % Custom item numbers for enumerations
\usepackage{hyperref} % For \autoref and hyperlinks
\usepackage{placeins} %DT$04/01/2020 \FloatBarrier 
\usepackage{color} % Required for custom colors
\usepackage{amsmath,amsfonts,stmaryrd,amssymb,theorem} % Math packages
\usepackage{siunitx} %DT$03/01/2021 symbols ex:\SI{50}{\micro\second}
\sisetup{range-phrase={\text{\ -\ }},
     input-decimal-markers={.}, 
     range-units = single,
     output-decimal-marker = {.},
     group-digits=false}

\usepackage{booktabs}

\newcommand{\reffig}[1]{Figure~\ref{#1}}
\newcommand{\reftab}[1]{Table~\ref{#1}}
\newcommand{\refeqn}[1]{Equation~\ref{#1}}
\NewDocumentCommand\mat{mmmm}{%
\text{$\begin{pmatrix}#1 & #2\\#3 & #4\end{pmatrix}$}%
}

%%%%%%%%%%%%%%%%%%%%%%%%%%%%%%%%%%%%%%%%%%%% COMMENTS COLORS %%%%%%%%%%%%%%%%%%%%%%%%%%%%%%%%%%%%%%%%
\usepackage{dsfont}
\usepackage{xcolor}
\usepackage{todonotes}
\usepackage{ulem}
\usepackage{listings}

% global

\definecolor{todoxcolor}{HTML}{11aa00}
\newcommand{\todox}[2]{{\color{todoxcolor}\sout{#1}#2}}
\renewcommand{\todox}[1]{\todo[color=todoxcolor!20, inline]{{\it TODO:} #1}}

\definecolor{UniPDcolor}{HTML}{9B0014}
\newcommand{\UniPD}[1]{\todo[color=UniPDcolor, inline]{{\color{white}{\it UniPD contribution:} #1}}}
\newcommand{\todoUPD}[1]{\todo[color=UniPDcolor!20, inline]{{\it TODO UniPD:} #1}}


% people

\definecolor{fedecolor}{HTML}{FEDEBE}
\definecolor{fedecolor2}{HTML}{ff7600}
\newcommand{\fede}[2]{{\color{fedecolor2}{(Fede:\sout{#1} #2})}}
\newcommand{\fedecom}[1]{\todo[color=fedecolor!40, inline]{{\it Fede:} #1}}